% Options for packages loaded elsewhere
\PassOptionsToPackage{unicode}{hyperref}
\PassOptionsToPackage{hyphens}{url}
%
\documentclass[
]{article}
\usepackage{amsmath,amssymb}
\usepackage{iftex}
\ifPDFTeX
  \usepackage[T1]{fontenc}
  \usepackage[utf8]{inputenc}
  \usepackage{textcomp} % provide euro and other symbols
\else % if luatex or xetex
  \usepackage{unicode-math} % this also loads fontspec
  \defaultfontfeatures{Scale=MatchLowercase}
  \defaultfontfeatures[\rmfamily]{Ligatures=TeX,Scale=1}
\fi
\usepackage{lmodern}
\ifPDFTeX\else
  % xetex/luatex font selection
\fi
% Use upquote if available, for straight quotes in verbatim environments
\IfFileExists{upquote.sty}{\usepackage{upquote}}{}
\IfFileExists{microtype.sty}{% use microtype if available
  \usepackage[]{microtype}
  \UseMicrotypeSet[protrusion]{basicmath} % disable protrusion for tt fonts
}{}
\makeatletter
\@ifundefined{KOMAClassName}{% if non-KOMA class
  \IfFileExists{parskip.sty}{%
    \usepackage{parskip}
  }{% else
    \setlength{\parindent}{0pt}
    \setlength{\parskip}{6pt plus 2pt minus 1pt}}
}{% if KOMA class
  \KOMAoptions{parskip=half}}
\makeatother
\usepackage{xcolor}
\usepackage[margin=1in]{geometry}
\usepackage{color}
\usepackage{fancyvrb}
\newcommand{\VerbBar}{|}
\newcommand{\VERB}{\Verb[commandchars=\\\{\}]}
\DefineVerbatimEnvironment{Highlighting}{Verbatim}{commandchars=\\\{\}}
% Add ',fontsize=\small' for more characters per line
\usepackage{framed}
\definecolor{shadecolor}{RGB}{248,248,248}
\newenvironment{Shaded}{\begin{snugshade}}{\end{snugshade}}
\newcommand{\AlertTok}[1]{\textcolor[rgb]{0.94,0.16,0.16}{#1}}
\newcommand{\AnnotationTok}[1]{\textcolor[rgb]{0.56,0.35,0.01}{\textbf{\textit{#1}}}}
\newcommand{\AttributeTok}[1]{\textcolor[rgb]{0.13,0.29,0.53}{#1}}
\newcommand{\BaseNTok}[1]{\textcolor[rgb]{0.00,0.00,0.81}{#1}}
\newcommand{\BuiltInTok}[1]{#1}
\newcommand{\CharTok}[1]{\textcolor[rgb]{0.31,0.60,0.02}{#1}}
\newcommand{\CommentTok}[1]{\textcolor[rgb]{0.56,0.35,0.01}{\textit{#1}}}
\newcommand{\CommentVarTok}[1]{\textcolor[rgb]{0.56,0.35,0.01}{\textbf{\textit{#1}}}}
\newcommand{\ConstantTok}[1]{\textcolor[rgb]{0.56,0.35,0.01}{#1}}
\newcommand{\ControlFlowTok}[1]{\textcolor[rgb]{0.13,0.29,0.53}{\textbf{#1}}}
\newcommand{\DataTypeTok}[1]{\textcolor[rgb]{0.13,0.29,0.53}{#1}}
\newcommand{\DecValTok}[1]{\textcolor[rgb]{0.00,0.00,0.81}{#1}}
\newcommand{\DocumentationTok}[1]{\textcolor[rgb]{0.56,0.35,0.01}{\textbf{\textit{#1}}}}
\newcommand{\ErrorTok}[1]{\textcolor[rgb]{0.64,0.00,0.00}{\textbf{#1}}}
\newcommand{\ExtensionTok}[1]{#1}
\newcommand{\FloatTok}[1]{\textcolor[rgb]{0.00,0.00,0.81}{#1}}
\newcommand{\FunctionTok}[1]{\textcolor[rgb]{0.13,0.29,0.53}{\textbf{#1}}}
\newcommand{\ImportTok}[1]{#1}
\newcommand{\InformationTok}[1]{\textcolor[rgb]{0.56,0.35,0.01}{\textbf{\textit{#1}}}}
\newcommand{\KeywordTok}[1]{\textcolor[rgb]{0.13,0.29,0.53}{\textbf{#1}}}
\newcommand{\NormalTok}[1]{#1}
\newcommand{\OperatorTok}[1]{\textcolor[rgb]{0.81,0.36,0.00}{\textbf{#1}}}
\newcommand{\OtherTok}[1]{\textcolor[rgb]{0.56,0.35,0.01}{#1}}
\newcommand{\PreprocessorTok}[1]{\textcolor[rgb]{0.56,0.35,0.01}{\textit{#1}}}
\newcommand{\RegionMarkerTok}[1]{#1}
\newcommand{\SpecialCharTok}[1]{\textcolor[rgb]{0.81,0.36,0.00}{\textbf{#1}}}
\newcommand{\SpecialStringTok}[1]{\textcolor[rgb]{0.31,0.60,0.02}{#1}}
\newcommand{\StringTok}[1]{\textcolor[rgb]{0.31,0.60,0.02}{#1}}
\newcommand{\VariableTok}[1]{\textcolor[rgb]{0.00,0.00,0.00}{#1}}
\newcommand{\VerbatimStringTok}[1]{\textcolor[rgb]{0.31,0.60,0.02}{#1}}
\newcommand{\WarningTok}[1]{\textcolor[rgb]{0.56,0.35,0.01}{\textbf{\textit{#1}}}}
\usepackage{graphicx}
\makeatletter
\def\maxwidth{\ifdim\Gin@nat@width>\linewidth\linewidth\else\Gin@nat@width\fi}
\def\maxheight{\ifdim\Gin@nat@height>\textheight\textheight\else\Gin@nat@height\fi}
\makeatother
% Scale images if necessary, so that they will not overflow the page
% margins by default, and it is still possible to overwrite the defaults
% using explicit options in \includegraphics[width, height, ...]{}
\setkeys{Gin}{width=\maxwidth,height=\maxheight,keepaspectratio}
% Set default figure placement to htbp
\makeatletter
\def\fps@figure{htbp}
\makeatother
\setlength{\emergencystretch}{3em} % prevent overfull lines
\providecommand{\tightlist}{%
  \setlength{\itemsep}{0pt}\setlength{\parskip}{0pt}}
\setcounter{secnumdepth}{-\maxdimen} % remove section numbering
\ifLuaTeX
  \usepackage{selnolig}  % disable illegal ligatures
\fi
\IfFileExists{bookmark.sty}{\usepackage{bookmark}}{\usepackage{hyperref}}
\IfFileExists{xurl.sty}{\usepackage{xurl}}{} % add URL line breaks if available
\urlstyle{same}
\hypersetup{
  pdftitle={cvicenie2.R},
  pdfauthor={matus},
  hidelinks,
  pdfcreator={LaTeX via pandoc}}

\title{cvicenie2.R}
\author{matus}
\date{2024-02-19}

\begin{document}
\maketitle

\begin{Shaded}
\begin{Highlighting}[]
\CommentTok{\#title:"cvicenie 2"}
\CommentTok{\#meno:"Norbert Matuska"}
\CommentTok{\#datum: 19.02.2024}

\CommentTok{\#nastavenie adresara}
\FunctionTok{setwd}\NormalTok{(}\StringTok{"C:}\SpecialCharTok{\textbackslash{}\textbackslash{}}\StringTok{Users}\SpecialCharTok{\textbackslash{}\textbackslash{}}\StringTok{matus}\SpecialCharTok{\textbackslash{}\textbackslash{}}\StringTok{Desktop}\SpecialCharTok{\textbackslash{}\textbackslash{}}\StringTok{PAS"}\NormalTok{)}
\FunctionTok{getwd}\NormalTok{()}
\end{Highlighting}
\end{Shaded}

\begin{verbatim}
## [1] "C:/Users/matus/Desktop/PAS"
\end{verbatim}

\begin{Shaded}
\begin{Highlighting}[]
\FunctionTok{library}\NormalTok{(Matrix)}
\CommentTok{\# sqrt() vie robit iba druhu odmocninu}
\NormalTok{a}\OtherTok{\textless{}{-}}\DecValTok{3}
\FunctionTok{class}\NormalTok{(a)}
\end{Highlighting}
\end{Shaded}

\begin{verbatim}
## [1] "numeric"
\end{verbatim}

\begin{Shaded}
\begin{Highlighting}[]
\NormalTok{b}\OtherTok{\textless{}{-}}\NormalTok{4L }\CommentTok{\# velke L je integer}
\FunctionTok{class}\NormalTok{(b)}
\end{Highlighting}
\end{Shaded}

\begin{verbatim}
## [1] "integer"
\end{verbatim}

\begin{Shaded}
\begin{Highlighting}[]
\NormalTok{z}\OtherTok{\textless{}{-}}\DecValTok{2}\SpecialCharTok{+}\NormalTok{3i}
\FunctionTok{class}\NormalTok{(z)}
\end{Highlighting}
\end{Shaded}

\begin{verbatim}
## [1] "complex"
\end{verbatim}

\begin{Shaded}
\begin{Highlighting}[]
\CommentTok{\# zmena datoveho typu}
\NormalTok{d}\OtherTok{\textless{}{-}}\FunctionTok{as.integer}\NormalTok{(a)}
\FunctionTok{is.integer}\NormalTok{(d)}
\end{Highlighting}
\end{Shaded}

\begin{verbatim}
## [1] TRUE
\end{verbatim}

\begin{Shaded}
\begin{Highlighting}[]
\NormalTok{d}
\end{Highlighting}
\end{Shaded}

\begin{verbatim}
## [1] 3
\end{verbatim}

\begin{Shaded}
\begin{Highlighting}[]
\CommentTok{\# logicke operatory}
\DecValTok{3}\SpecialCharTok{\textgreater{}}\DecValTok{7}
\end{Highlighting}
\end{Shaded}

\begin{verbatim}
## [1] FALSE
\end{verbatim}

\begin{Shaded}
\begin{Highlighting}[]
\DecValTok{5}\SpecialCharTok{==}\DecValTok{5}
\end{Highlighting}
\end{Shaded}

\begin{verbatim}
## [1] TRUE
\end{verbatim}

\begin{Shaded}
\begin{Highlighting}[]
\DecValTok{4}\SpecialCharTok{!=}\DecValTok{5}
\end{Highlighting}
\end{Shaded}

\begin{verbatim}
## [1] TRUE
\end{verbatim}

\begin{Shaded}
\begin{Highlighting}[]
\CommentTok{\# Data structures}
\NormalTok{v1}\OtherTok{\textless{}{-}}\FunctionTok{c}\NormalTok{(}\DecValTok{2}\NormalTok{,}\DecValTok{3}\NormalTok{,}\DecValTok{4}\NormalTok{)}
\NormalTok{v2}\OtherTok{\textless{}{-}}\FunctionTok{c}\NormalTok{(}\DecValTok{1}\NormalTok{,}\DecValTok{0}\NormalTok{,}\DecValTok{2}\NormalTok{)}
\NormalTok{v1}\SpecialCharTok{+}\NormalTok{v2 }\CommentTok{\#scitanie vektorov}
\end{Highlighting}
\end{Shaded}

\begin{verbatim}
## [1] 3 3 6
\end{verbatim}

\begin{Shaded}
\begin{Highlighting}[]
\NormalTok{v1}\SpecialCharTok{*}\NormalTok{v2}
\end{Highlighting}
\end{Shaded}

\begin{verbatim}
## [1] 2 0 8
\end{verbatim}

\begin{Shaded}
\begin{Highlighting}[]
\NormalTok{v1}\SpecialCharTok{\%*\%}\NormalTok{v2 }\CommentTok{\#skalarne nasobanie}
\end{Highlighting}
\end{Shaded}

\begin{verbatim}
##      [,1]
## [1,]   10
\end{verbatim}

\begin{Shaded}
\begin{Highlighting}[]
\NormalTok{v1}\SpecialCharTok{*}\FunctionTok{c}\NormalTok{(}\DecValTok{1}\NormalTok{,}\DecValTok{2}\NormalTok{) }\CommentTok{\#treba davat bacha na dlzku}
\end{Highlighting}
\end{Shaded}

\begin{verbatim}
## Warning in v1 * c(1, 2): longer object length is not a multiple of shorter
## object length
\end{verbatim}

\begin{verbatim}
## [1] 2 6 4
\end{verbatim}

\begin{Shaded}
\begin{Highlighting}[]
\FunctionTok{length}\NormalTok{(v1)}
\end{Highlighting}
\end{Shaded}

\begin{verbatim}
## [1] 3
\end{verbatim}

\begin{Shaded}
\begin{Highlighting}[]
\NormalTok{v3}\OtherTok{\textless{}{-}}\DecValTok{2}\SpecialCharTok{:}\DecValTok{24} \CommentTok{\# vektor zadany postupnostou, cize od 2 po 24 naplni po kroku 1}
\NormalTok{v3}
\end{Highlighting}
\end{Shaded}

\begin{verbatim}
##  [1]  2  3  4  5  6  7  8  9 10 11 12 13 14 15 16 17 18 19 20 21 22 23 24
\end{verbatim}

\begin{Shaded}
\begin{Highlighting}[]
\NormalTok{v4}\OtherTok{\textless{}{-}}\FunctionTok{seq}\NormalTok{(}\AttributeTok{from=}\DecValTok{2}\NormalTok{,}\AttributeTok{to=}\DecValTok{24}\NormalTok{,}\AttributeTok{by=}\DecValTok{3}\NormalTok{)}
\NormalTok{v4}
\end{Highlighting}
\end{Shaded}

\begin{verbatim}
## [1]  2  5  8 11 14 17 20 23
\end{verbatim}

\begin{Shaded}
\begin{Highlighting}[]
\NormalTok{v5}\OtherTok{\textless{}{-}}\FunctionTok{rep}\NormalTok{(v1,}\AttributeTok{times=}\DecValTok{4}\NormalTok{) }\CommentTok{\# zreplikuj vektor1 4x  za sebou}
\NormalTok{v5}
\end{Highlighting}
\end{Shaded}

\begin{verbatim}
##  [1] 2 3 4 2 3 4 2 3 4 2 3 4
\end{verbatim}

\begin{Shaded}
\begin{Highlighting}[]
\FunctionTok{rep}\NormalTok{(v1,}\AttributeTok{times=}\FunctionTok{c}\NormalTok{(}\DecValTok{3}\NormalTok{,}\DecValTok{2}\NormalTok{,}\DecValTok{5}\NormalTok{)) }\CommentTok{\# kolkokrat sa bude dana suradnica opakovat}
\end{Highlighting}
\end{Shaded}

\begin{verbatim}
##  [1] 2 2 2 3 3 4 4 4 4 4
\end{verbatim}

\begin{Shaded}
\begin{Highlighting}[]
\FunctionTok{sum}\NormalTok{(v4) }\CommentTok{\# sucet vsetkych prvkov vo vektore}
\end{Highlighting}
\end{Shaded}

\begin{verbatim}
## [1] 100
\end{verbatim}

\begin{Shaded}
\begin{Highlighting}[]
\FunctionTok{prod}\NormalTok{(v1) }\CommentTok{\# sucin vsetkych prvkov vo vektore}
\end{Highlighting}
\end{Shaded}

\begin{verbatim}
## [1] 24
\end{verbatim}

\begin{Shaded}
\begin{Highlighting}[]
\FunctionTok{sum}\NormalTok{(v4[}\DecValTok{3}\SpecialCharTok{:}\DecValTok{7}\NormalTok{])}
\end{Highlighting}
\end{Shaded}

\begin{verbatim}
## [1] 70
\end{verbatim}

\begin{Shaded}
\begin{Highlighting}[]
\FunctionTok{cumsum}\NormalTok{(v1) }\CommentTok{\# kumulovany sucet UwU}
\end{Highlighting}
\end{Shaded}

\begin{verbatim}
## [1] 2 5 9
\end{verbatim}

\begin{Shaded}
\begin{Highlighting}[]
\CommentTok{\# matice}
\NormalTok{v6}\OtherTok{\textless{}{-}}\NormalTok{v3[}\DecValTok{1}\SpecialCharTok{:}\DecValTok{12}\NormalTok{]}
\NormalTok{v6}
\end{Highlighting}
\end{Shaded}

\begin{verbatim}
##  [1]  2  3  4  5  6  7  8  9 10 11 12 13
\end{verbatim}

\begin{Shaded}
\begin{Highlighting}[]
\NormalTok{A}\OtherTok{\textless{}{-}}\FunctionTok{Matrix}\NormalTok{(v6,}\DecValTok{4}\NormalTok{,}\DecValTok{3}\NormalTok{) }\CommentTok{\# matica z v6 vektora zapisana na 4x3 maticu}
\NormalTok{A}
\end{Highlighting}
\end{Shaded}

\begin{verbatim}
## 4 x 3 Matrix of class "dgeMatrix"
##      [,1] [,2] [,3]
## [1,]    2    6   10
## [2,]    3    7   11
## [3,]    4    8   12
## [4,]    5    9   13
\end{verbatim}

\begin{Shaded}
\begin{Highlighting}[]
\NormalTok{B}\OtherTok{\textless{}{-}}\FunctionTok{Matrix}\NormalTok{(v6,}\DecValTok{4}\NormalTok{,}\DecValTok{3}\NormalTok{,}\AttributeTok{byrow=}\NormalTok{T)}
\NormalTok{B}
\end{Highlighting}
\end{Shaded}

\begin{verbatim}
## 4 x 3 Matrix of class "dgeMatrix"
##      [,1] [,2] [,3]
## [1,]    2    3    4
## [2,]    5    6    7
## [3,]    8    9   10
## [4,]   11   12   13
\end{verbatim}

\begin{Shaded}
\begin{Highlighting}[]
\CommentTok{\#transponovana matica}
\FunctionTok{t}\NormalTok{(B)}
\end{Highlighting}
\end{Shaded}

\begin{verbatim}
## 3 x 4 Matrix of class "dgeMatrix"
##      [,1] [,2] [,3] [,4]
## [1,]    2    5    8   11
## [2,]    3    6    9   12
## [3,]    4    7   10   13
\end{verbatim}

\begin{Shaded}
\begin{Highlighting}[]
\CommentTok{\#vyber prvku z matice}
\NormalTok{B[}\DecValTok{2}\NormalTok{,}\DecValTok{3}\NormalTok{]}
\end{Highlighting}
\end{Shaded}

\begin{verbatim}
## [1] 7
\end{verbatim}

\begin{Shaded}
\begin{Highlighting}[]
\CommentTok{\#vyber podpola, podmatice}
\NormalTok{B[}\DecValTok{1}\SpecialCharTok{:}\DecValTok{2}\NormalTok{,}\DecValTok{2}\SpecialCharTok{:}\DecValTok{3}\NormalTok{]}
\end{Highlighting}
\end{Shaded}

\begin{verbatim}
## 2 x 2 Matrix of class "dgeMatrix"
##      [,1] [,2]
## [1,]    3    4
## [2,]    6    7
\end{verbatim}

\begin{Shaded}
\begin{Highlighting}[]
\CommentTok{\#data.frame, datovy ramec}
\NormalTok{meno}\OtherTok{=}\FunctionTok{c}\NormalTok{(}\StringTok{"Ala"}\NormalTok{,}\StringTok{"Jojo"}\NormalTok{,}\StringTok{"Jana"}\NormalTok{,}\StringTok{"Palo"}\NormalTok{,}\StringTok{"Miro"}\NormalTok{,}\StringTok{"Eva"}\NormalTok{)}
\NormalTok{vek}\OtherTok{=}\FunctionTok{c}\NormalTok{(}\DecValTok{14}\NormalTok{,}\DecValTok{16}\NormalTok{,}\DecValTok{14}\NormalTok{,}\DecValTok{17}\NormalTok{,}\DecValTok{15}\NormalTok{,}\DecValTok{13}\NormalTok{)}
\NormalTok{deti}\OtherTok{=}\FunctionTok{data.frame}\NormalTok{(meno,vek)}
\NormalTok{deti}
\end{Highlighting}
\end{Shaded}

\begin{verbatim}
##   meno vek
## 1  Ala  14
## 2 Jojo  16
## 3 Jana  14
## 4 Palo  17
## 5 Miro  15
## 6  Eva  13
\end{verbatim}

\begin{Shaded}
\begin{Highlighting}[]
\CommentTok{\#pridanie parametra do datoveho ramca}
\NormalTok{deti}\SpecialCharTok{$}\NormalTok{test}\OtherTok{=}\FunctionTok{c}\NormalTok{(}\DecValTok{10}\NormalTok{,}\DecValTok{11}\NormalTok{,}\DecValTok{15}\NormalTok{,}\DecValTok{14}\NormalTok{,}\DecValTok{12}\NormalTok{,}\DecValTok{10}\NormalTok{)}
\NormalTok{deti}
\end{Highlighting}
\end{Shaded}

\begin{verbatim}
##   meno vek test
## 1  Ala  14   10
## 2 Jojo  16   11
## 3 Jana  14   15
## 4 Palo  17   14
## 5 Miro  15   12
## 6  Eva  13   10
\end{verbatim}

\begin{Shaded}
\begin{Highlighting}[]
\CommentTok{\#pridanie neciselneho parametra a pouzitie prikazu faktor}
\NormalTok{deti}\SpecialCharTok{$}\NormalTok{pohlavie}\OtherTok{=}\FunctionTok{factor}\NormalTok{(}\FunctionTok{c}\NormalTok{(}\DecValTok{1}\NormalTok{,}\DecValTok{0}\NormalTok{,}\DecValTok{1}\NormalTok{,}\DecValTok{0}\NormalTok{,}\DecValTok{0}\NormalTok{,}\DecValTok{1}\NormalTok{), }\AttributeTok{labels=}\FunctionTok{c}\NormalTok{(}\StringTok{"m"}\NormalTok{,}\StringTok{"z"}\NormalTok{))}
\NormalTok{deti}
\end{Highlighting}
\end{Shaded}

\begin{verbatim}
##   meno vek test pohlavie
## 1  Ala  14   10        z
## 2 Jojo  16   11        m
## 3 Jana  14   15        z
## 4 Palo  17   14        m
## 5 Miro  15   12        m
## 6  Eva  13   10        z
\end{verbatim}

\begin{Shaded}
\begin{Highlighting}[]
\CommentTok{\#filtrovanie dat}
\NormalTok{zeny}\OtherTok{\textless{}{-}}\FunctionTok{subset}\NormalTok{(deti,deti}\SpecialCharTok{$}\NormalTok{pohlavie}\SpecialCharTok{==}\StringTok{"z"}\NormalTok{)}
\NormalTok{zeny}
\end{Highlighting}
\end{Shaded}

\begin{verbatim}
##   meno vek test pohlavie
## 1  Ala  14   10        z
## 3 Jana  14   15        z
## 6  Eva  13   10        z
\end{verbatim}

\begin{Shaded}
\begin{Highlighting}[]
\FunctionTok{min}\NormalTok{(deti}\SpecialCharTok{$}\NormalTok{vek)}
\end{Highlighting}
\end{Shaded}

\begin{verbatim}
## [1] 13
\end{verbatim}

\begin{Shaded}
\begin{Highlighting}[]
\FunctionTok{max}\NormalTok{(deti}\SpecialCharTok{$}\NormalTok{test)}
\end{Highlighting}
\end{Shaded}

\begin{verbatim}
## [1] 15
\end{verbatim}

\begin{Shaded}
\begin{Highlighting}[]
\FunctionTok{mean}\NormalTok{(deti}\SpecialCharTok{$}\NormalTok{test) }\CommentTok{\#aritmeticky priemer z testu}
\end{Highlighting}
\end{Shaded}

\begin{verbatim}
## [1] 12
\end{verbatim}

\begin{Shaded}
\begin{Highlighting}[]
\FunctionTok{var}\NormalTok{(deti}\SpecialCharTok{$}\NormalTok{test) }\CommentTok{\#odchylka kvadrantu, o kolko sa v kvadrante deti lisili}
\end{Highlighting}
\end{Shaded}

\begin{verbatim}
## [1] 4.4
\end{verbatim}

\begin{Shaded}
\begin{Highlighting}[]
\FunctionTok{summary}\NormalTok{(deti)}
\end{Highlighting}
\end{Shaded}

\begin{verbatim}
##      meno                vek             test       pohlavie
##  Length:6           Min.   :13.00   Min.   :10.00   m:3     
##  Class :character   1st Qu.:14.00   1st Qu.:10.25   z:3     
##  Mode  :character   Median :14.50   Median :11.50           
##                     Mean   :14.83   Mean   :12.00           
##                     3rd Qu.:15.75   3rd Qu.:13.50           
##                     Max.   :17.00   Max.   :15.00
\end{verbatim}

\end{document}
